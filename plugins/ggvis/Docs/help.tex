\documentclass[11pt]{article}
\usepackage{fullpage}

\newcommand{\Reals}   {{\rm I \! R}}
\newcommand{\vx}      {{\mbox{\bf x}}}

\begin{document}

\title{ggvis help}
\author{Andreas Buja and Deborah F. Swayne}
\maketitle

% Background

\section{Introduction}

ggvis uses multidimensional scaling (MDS) to map a set of objects
to points in $k$-space such that

  dissimilarity( object $i$, object $~$ distance( point $i$, point $j$ )

This is made precise by minimizing the chosen stress function.

The parameters of the function are:
\begin{itemize}
\item   

$D ~ =$ target distance matrix 
    This is either supplied as a matrix of dissimilarities or
    distances, or it uses the shortest-path metric if a discrete
    graph is given in terms of a links file.
\item
$p$ = exponent of the power transformation of $D$
\item
$l ~ =$ Minkowski norm used to calculate the current interpoint 
    distances ($l=2$ corresponds to the Euclidean metric on $k$-space)
\item
$r ~ =$ residual power applied to the difference between the target
    distance $D$ and the interpoint distances
\item
$T ~ =$ threshold above which dissimilarities are removed from the 
    stress function
\end{itemize}

The value of the stress function is plotted as a number generally between
zero and one.

Below the plot of the stress function is a barplot of the distribution
of $D^p$.  As you change p, this plot is updated.  Use the left mouse
button to set the threshold T used in the stress function: the solid
bars are below the threshold and represent the distances used in MDS;
the hollow bars correspond to pairs of points that are ignored.

Other parameters available on the control panel are
\begin{itemize}
\item
Dimension (k): the dimensionality of the space into which the data are
  being mapped.  The default is 3D.  Examine your point configurations
  with 3D rotations or grand tours in ggobi.
\item
Stepsize: the size of the step taken to bring $x_i$ and $x_j$ closer
  together or to push them farther apart.  You might start with a
  large stepsize and then reduce it once the structure begins to
  stabilize.  You can also increase it suddenly in a form of
  interactive "simulated annealing," to jump out of a local minimum
  when you think there might be a lower minimum elsewhere.
\item
MDS within subgroups: If you have subgroups in the data that you would
  like to handle separately from one another, then brush them using
  ggobi.  Then select this button and restart "Run MDS."  Now only
  distances between points with the same glyph and color will be
  considered.
\item
MDS casewise: If you would like to investigate the behavior of a
  single case, or a small group of cases, then use ggobi to add
  persistent labels to those cases.  Then select this button and
  restart "Run MDS."  Now only distances between pairs of points
  including one of the labelled points will be considered.  (This
  feature can be used in conjunction with brush groups.  Then only
  distances between pairs of points with the same color and glyph AND
  within the same brush group will be considered.)
\item
Shepard diagram:  Launch ggobi display containing a plot of $D^p$ vs
  the configuration distances.  
%   You can first specify that you
%   would like to open the display with an option:  its initial display
%   can be a random sample of the data.  That is, all the data will
%   be written out to a file (configd0.dat, configd1.dat, ...) and
%   then read in by ggobi, but a subset may be chosen for display.
%   (See Subset Data on the ggobi Tools Menu.)
\end{itemize}

%The input file can contain missing values, represented by the strings
%"na" or "NA" or "."  By making use of specific missing value patterns,
%ggvis can be used to perform "multidimensional unfolding."

\section{Optimization of the configuration}

% Rewrite this section which describes the initial steps --
% specify the datasets, define D, and run
Since you were able to start up ggvis, begin by
clicking "Run MDS" and see what happens: You should see the points in
the ggobi window move and a curve evolve in a subwindow titled "Stress
function" in the ggvis panel.  The motion is generated by a gradient
descent algorithm; the curve shows the criterion, called Stress.

Click again and the motion stops.  Click one more time and leave "Run
MDS" on in what follows.  This allows you to immediately see the
effects of various manipulations.

By default, ggvis performs gradient descent of a stress function for
metric Kruskal-Shepard distance scaling.
% link to the formula

\section{Controls in the ggvis panel}

\subsection{First tabbed widget: Specify datasets, define $D$, and Run}

\begin{itemize}
\item
k = Dimension of the points, default: 3.
    For 3D viewing, choose in the ggobi console ViewMode $\rightarrow$ Rotation
    For 2D viewing, choose in the ggobi window ViewMode $\rightarrow$ XYPlot
    The dimension k can be as high as 12.  
    For viewing 4D and higher, choose in ggobi ViewMode $\rightarrow$ 2D Tour
\item
Stepsize: The stress optimization uses a fixed-size gradient step.
    This slider permits controlling the step size in terms of a
    fraction of the size of the point configuration.  Default: 0.02
    (gradient sz = 2\% config sz).
\item
Run MDS: Turn optimization on and off
\item
Step: an alternative to "Run MDS" if you wish to follow the gradient
   steps one by one.
\end{itemize}

\subsection{Plot of the stress function}

Stress function plot: shows the evolution of the stress values as
optimization progresses.  No interactive controls. 

\subsection{Key MDS parameters}

\begin{itemize}

\item
"Metric" versus "Nonmetric": choice of metric and nonmetric MDS
\item
"Krsk/Sh" versus "Classic": choice of Kruskal-Shepard distance scaling
   versus Torgersen-Gower dot-product scaling (also called classical MDS).
% link to the formulas at the end of the document
\end{itemize}

IMPORTANT for those who know about MDS: ggvis has a version of
   nonmetric classical (!) scaling.  That is, the two pairs of choices
   above are in fact fully crossed.  Nonmetric classical scaling is
   not very stable, but it works with good starting configurations.

   [Technical remark: The idea is to perform dot-product scaling of
   $-D^2$ under the constraint that the configuration remains centered.
   This is equivalent to doubly-centering $-D^2$ and performing
   unconstrained dot-product scaling.  The advantage of the former is
   that it can be made nonmetric by replacing $-D^2$ with an isotonic
   transform $f(-D)$ which can be estimated by regressing the
   configuration distances on$ -D$.]

\subsection{Data ($D$): Histogram, power, weights}

\begin{itemize}
\item
Histogram of transformed dissimilarities (bottom of the ggvis panel):
   Note the grips on either end of the horizontal axis!  You can move
   them to trim large and small dissimilarities.  The right grip trims
   large ones; the left grip small ones.  This allows you to check
   their importance.
\item
p = Exponent of the power transformation of D in metric MDS.
    Playing with the slider allows you to interactively optimize stress.
    Also, see what happens as $p$ $\rightarrow$ 0 (scaling of a simplex).
\item
r = Weight parameter: The dissimilarities D can be weighted with $w=D^r$
    in the stress function.  Default: $r=0$, equal weights.
\end{itemize}

\subsection{Menubar: View, Reset}

\begin{itemize}

\item View

"Shepard Plot": starts up a separate ggobi window with the following
   variables:
\begin{itemize}
\item
   $d_{ij}$     = interpoint distances
\item
   $f(D_{ij})$  = transformed dissimilarities; 
              metric: power; nonmetric: isotonic.
\item
   $D_{ij}$     = raw dissimilarities
\item
   $Res_{ij}$   = residuals $f(D_{ij})$ - $d_{ij}$
\item
   $Wgt_{ij}$   = weights, informative only if $r$ or $wb$ is non-default
   i        = the first index  \   useful for checking the patterns
   j        = the second index /   of missing or omitted dissimilarities
\end{itemize}
              
   The number of dissimilarities can be large!  The field below
   "Shepard Plot" tells you how many there are.  For N>500 you might
   want to be careful.  Use "Select'n prob" with "Run MDS" off to get
   a random sample of manageable size.

\item Reset
  \begin{itemize}
  \item
  Reinit layout: Re-initialize the configuration to the initial positions.
  \item
  Scramble layout
  \item
  Reinit parameters

%"Center config": Center and scale the configuration.  Can be useful if
%  for some reason the configuration has collapsed or got lost from the
%  window.

  \end{itemize}

\subsection{Final tabbed widget: distance, groups, sensitivity, constraints}

begin{itemize}

\item Distance
\begin{itemize}
\item
$p$ = exponent of the power transformation of $D$
\item
m = Minkowski norm: Used to calculate the current interpoint distances.
    Default: $m=2$, the Euclidean metric on k-space.
\end{itemize}


\item Groups
% This section needs restructuring

  \begin{itemize}
  \item
    Click "Groups", a window for group selection will pop up.  Click
    in the "Hide" column to remove and restore groups.  Watch the
    effects on the point configuration.
  \item
    Click on the menu underneath "MDS with Groups".  
    \begin{itemize}
    \item
    Select "Dists within groups": same as wb=(2,0) above, but
    implemented more efficiently without weights.
    Similar for "Dists between groups" (~ wb=(0,2)).
    Default: "Ignore groups" (~ wb=(1,1)).
    \item
    Select "Dists from\&in anchor": anchored MDS maps points within an
    anchor set by MDS (as usual), but it maps the points outside the
    anchor set with the dissimilarities w.r.t. the anchor set (hence
    the term).

    The anchor set consists of the points with the glyph that appears
    to the right of "MDS with Groups".  You can change the anchor set
    by selecting "View:..." -> "Move Points" in the ggobi window.
    Click MIDDLE on a point, and the points with that color glyph
    become the new anchor set.
    \item
    Select "Dists from anchor": same, except the anchor points remain
    fixed.  This allows you to mess with the anchor points manually by
    dragging them around with the LEFT mouse button depressed.
    \end{itemize}
\item
wb = Within/between parameter: If there are color/glyph groups, the
    parameter allows one to differentially weight dissimilarities that
    link points within groups versus between groups.  
    Extremes: wb=(2,0): Use only within groups, remove between groups;
              wb=(0,2): Remove within groups, use only between groups.
    Default:  wb=(1,1): Weight within and between groups equally.    

    Color/glyph groups can be provided in input
    or created interactively with ggobi's Brush mode.
  \end{itemize}

\item Sensitivity
\begin{itemize}
\item
Selection probability: random subselection of dissimilarities; a form
   of stability check.  Implementation: a uniform [0,1] random number
   is generated for each dissimilarity, which is included in the
   stress function if the random number is below the selection
   probability read from the slider.  The "New" button allows you to
   create a new set of random numbers.  Select for example a selection
   probability 0.9 and click "New" repeatedly.  This will show how
   stable the configuration is under removal of roughly 1 in 10
   dissimilarities.  Default: 1.0 (all dissimilarities included).
\item
Perturbation: perturb the configuration with normal random vectors.
   The slider value s determines the relative fractions of
   configuration and perturbation: (1-s)*Config + s*Random.  
   Extremes: s=1.0: pure random, i.e., a random start (default).  
             s=0.0: pure configuration, i.e., no perturbation.
   Click "New" to initiate a perturbation.
\end{itemize}

\item Constraints

\end{itemize}

%"File": following conventions, this menu has I/O and Exit.
%   You can save the dissimilarity matrix, which can be useful if it
%   was generated from a discrete graph or a multivariate dataset.


\section{Formulas}

% formula for Kruskal-Shepard distance scaling

\begin{center}
{\Large \bf Kruskal-Shepard Distance Scaling}
\end{center}

\begin{eqnarray*}
&& {\rm STRESS}_D(\vx_1,...,\vx_N) ~=~ \left( 1 - \cos^2 \right)^{1/2}
\\
&& ~~~~~~~~ 
\cos^2 ~=~ 
    \frac{ 
      \left( \; 
        \sum_{ (i,j) \in I } ~
          w_{i,j} \cdot f(D_{i,j}) \cdot {\| \vx_i - \vx_j \|}_m^q
      \; \right)^2
    }{
      \left( \sum_{ (i,j) \in I } ~w_{i,j} \cdot f(D_{i,j})^2 \right) 
      \left( \sum_{ (i,j) \in I } ~w_{i,j} \cdot {\| \vx_i - \vx_j \|}_m^2 \right)
    }
\\
~
\\
~
\\
&& D_{i,j} \in \Reals,~ \ge 0, ~ N \times N ~ {\rm matrix~of~dissimilarity~data}
\\
~
\\
&& f(D_{i,j}) ~=~ \left\{ \begin{array}{ll}
                  D_{i,j}^p~,               ~~~& {\rm for~metric~MDS}
                  \\
                  {\rm Isotonic}(D_{i,j})~, ~~~& {\rm for~nonmetric~MDS}
                  \end{array} \right.
\\
&& ~~~~~~~~~~~~~~~
\begin{array}{l}
0 \le p \le 6, ~{\rm default:}~1~({\rm no~transformation}) \\
{\rm Isotonic} = {\rm monotone~transformation~estimated} \\
{\rm ~~~~~~~~~~~~~~ with~isotonic~regression}
\end{array}
\\
~
\\
&& \vx_1,...,\vx_N \in \Reals^k, ~ {\rm configuration~points} 
\\
&& ~~~~~~~~~~~~~~~~~
1 \le k \le 12, ~{\rm default:}~3~({\rm use~3D~Rotations~or~Grand~Tour})
\\
~
\\
&& {\| \vx_i - \vx_j \|}_m^q ~=~ 
   ( \sum_{\nu=1,...,k} | x_{i,\nu} - x_{j,\nu} |^m )^{q/m},
~~~~~~
{\rm configuration~distances,}~(..)^q
\\
&& ~~~~~~~~~~~~~~~
\begin{array}{l}
1 \le m \le 6, ~{\rm default:}~2~({\rm Euclidean};~ m=1:~{\rm City block})
\\
0 \le q \le 6, ~{\rm default:}~1 ~({\rm STRESS};~ q=2:~{\rm SSTRESS})
\end{array}
\\
~
\\
&& 
w_{i,j} ~=~ D_{i,j}^r \cdot \left\{ \begin{array}{ll}
                            w     ~, ~~~~~& {\rm if~color/glyph~of}~i,j~{\rm is~same}      \\
                            (2-w) ~, ~~~~~& {\rm if~color/glyph~of}~i,j~{\rm is~different}
			    \end{array} \right. 
\\
&& ~~~~~~~~~~~~~~~
\begin{array}{l}
-4 \le r \le +4,  ~{\rm default:}~0 ~ ({\rm equal~weights};~ r=-1:~{\rm Sammons})
\\
0 \le w \le 2, ~{\rm default:}~1 ~ ({\rm ignore~grps};~w=2:~{\rm within};~w=0:~{\rm between~grps})
\end{array}
\\
~
\\
&& I ~=~ \{ \; (i,j) \; |~ i \neq j ,~D_{i,j} \neq NA ,~T_0 < D_{i,j} < T_1 ,~~{\rm Runif}(i,j)<\alpha ,~ ...~ \}~
\\
&& ~~~~~~~~~~~~~~~
\begin{array}{l}
0 \le T_0 \le T_1, ~{\rm thresholds,~defaults:}~0,~\infty          \\
{\rm Runif} = {\rm uniform~random~numbers} \in [ 0,1 ] \\
\alpha = {\rm selection~probability,~ default:}~1      \\
... = {\rm conditions~based~on~color/glyph~groups}
\end{array}
\end{eqnarray*}

% formula for Torgerson-Gower dot-product scaling (classic MDS)

\begin{center}
{\Large \bf Torgerson-Gower Dot-Product Scaling}
\end{center}

\begin{eqnarray*}
&& {\rm STRAIN}_D(\vx_1,...,\vx_N) ~=~ \left( 1 - \cos^2 \right)^{1/2}
\\
&& ~~~~~~~~ 
\cos^2 ~=~ 
    \frac{ 
      \left( \; 
        \sum_{ (i,j) \in I } ~
          w_{i,j} \cdot f(-D_{i,j}^2) \cdot {\langle \vx_i , \vx_j \rangle}
      \; \right)^2
    }{
      \left( \sum_{ (i,j) \in I } ~w_{i,j} \cdot f(-D_{i,j}^2) \right) 
      \left( \sum_{ (i,j) \in I } ~w_{i,j} \cdot {\langle \vx_i , \vx_j \rangle} \right)
    }
\\
~
\\
~
\\
&& D_{i,j} \in \Reals,~ \ge 0, ~ N \times N ~ {\rm matrix~of~dissimilarity~data}
\\
~
\\
&& f(-D_{i,j}^2) ~=~ \left\{ \begin{array}{ll}
                  -D_{i,j}^{2p}~,               ~~~& {\rm for~metric~MDS}
                  \\
                  {\rm Isotonic}(-D_{i,j})~, ~~~& {\rm for~nonmetric~MDS}
                  \end{array} \right.
\\
&& ~~~~~~~~~~~~~~~
\begin{array}{l}
0 \le p \le 6, ~{\rm default:}~1~({\rm no~transformation}) \\
{\rm Isotonic} = {\rm monotone~transformation~estimated} \\
{\rm ~~~~~~~~~~~~~~ with~isotonic~regression}
\end{array}
\\
~
\\
&& \vx_1,...,\vx_N \in \Reals^k, ~ {\rm configuration~points,~constrained~to} ~\sum \vx_i = 0
\\
&& ~~~~~~~~~~~~~~~~~
1 \le k \le 12, ~{\rm default:}~3~({\rm use~3D~Rotations~or~Grand~Tour})
\\
~
\\
&& {\langle \vx_i , \vx_j \rangle} ~=~ 
   \sum_{\nu=1,...,k}  x_{i,\nu} \cdot x_{j,\nu} ,
~~~~~~
{\rm configuration~dot~products,}
\\
~
\\
&& 
w_{i,j} ~=~ D_{i,j}^r \cdot \left\{ \begin{array}{ll}
                            w     ~, ~~~~~& {\rm if~color/glyph~of}~i,j~{\rm is~same}      \\
                            (2-w) ~, ~~~~~& {\rm if~color/glyph~of}~i,j~{\rm is~different}
			    \end{array} \right. 
\\
&& ~~~~~~~~~~~~~~~
\begin{array}{l}
-4 \le r \le +4,  ~{\rm default:}~0 ~ ({\rm equal~weights})
\\
0 \le w \le 2, ~{\rm default:}~1 ~ ({\rm ignore~grps};~w=2:~{\rm within};~w=0:~{\rm between~grps})
\end{array}
\\
~
\\
&& I ~=~ \{ \; (i,j) \; |~ i \neq j ,~D_{i,j} \neq NA ,~T_0 < D_{i,j} < T_1 ,~~{\rm Runif}(i,j)<\alpha ,~ ...~ \}~
\\
&& ~~~~~~~~~~~~~~~
\begin{array}{l}
0 \le T_0 \le T_1, ~{\rm thresholds,~defaults:}~0,~\infty          \\
{\rm Runif} = {\rm uniform~random~numbers} \in [ 0,1 ] \\
\alpha = {\rm selection~probability,~ default:}~1      \\
... = {\rm conditions~based~on~color/glyph~groups}
\end{array}
\end{eqnarray*}


\end{document}




