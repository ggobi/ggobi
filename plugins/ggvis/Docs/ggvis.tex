\documentclass[11pt]{article}
\usepackage{fullpage}

\input{WebMacros}
\input{SMacros}
\input{CMacros}
\input{XMLMacros}

\begin{document}

\title{ggvis: graph layout as a plugin in {\tt ggobi}}
\author{Deborah F. Swayne}
\date{}
\maketitle

\begin{abstract}

ggvis is a graph layout plugin for ggobi, and this manual
describes its use.  It can be expected to change and grow in the
coming months.

\end{abstract}

\section{The data}

First, you need some data.  The ascii format is not adequate
for the specification of edges, so you'll probably use an xml
file with at least two datasets:  a set of cases or nodes, and
a set of edges.  The records in the node data must have
\XMLAttribute{id}s:

\begin{verbatim}
<record id="0"> ... </record>
\end{verbatim}

The edge records use those \XMLAttribute{id}s to specify a
\XMLAttribute{source} and \XMLAttribute{destination}:

\begin{verbatim}
<record source="0" destination="2"> ... </record>
\end{verbatim}

The sample data that is discussed here is part of the ggobi
distribution: \file{snetwork.xml}, an artificial social network
of 140 people who are connected by 205 edges, representing
some form of social contact.  Notice that there are two variables
recorded for each node, and two variables for each edge.  

In a ggobi scatterplot display, use the {\bf Edges} menu to
display the edges, and maybe the ``arrowheads'' which indicate
edge direction.  (Until the edges are displayed, neither layout
algorithm will work.)

\section{Layout}

There are two layout algorithms presently implemented in ggvis,
classical multidimensional scaling (with an optional force
adjustment algorithm), and a radial layout.

\subsection {Multidimensional scaling}

\subsection {Radial layout}

\section{Interactions}

% One specifies a plugin using the usual XML \XMLTag{plugin} syntax in
% an initialization file for GGobi.  For an R plugin, one specifies a
% \XMLAttr{language} attribute with a value \verb+"R"+.  (One should
% declare the R (meta-) plugin itself before this.)  In addition to the
% usual \XMLTag{description} and \XMLTag{author} information, one
% provides values for the \XMLAttr{init} and \XMLAttr{create} attributes
% of the \XMLTag{plugin} tag.

\section{Animation via R}

% Bibliography:  Graham Wills layout paper, xgvis papers

\end{document}
