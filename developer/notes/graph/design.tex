\documentclass[11pt]{article}

\setlength{\topmargin}{-0.25in}
\setlength{\textheight}{9in}
\setlength{\textwidth}{7.25in}
\setlength{\oddsidemargin}{-.4in}

\begin{document}

\begin{center}
\LARGE{ Design of interactive graph visualization software }
\end{center}
\centerline {\today}

\parindent 0in
\parskip 6 pt

\section {The data}

We are interested in the interactive exploratory data analysis of
communications transaction data.  For these transactions, there
is always a source and destination, which can be telephone
numbers, email addresses, or IP addresses.  We probably have some
data on the vertices (biz/res indicator, whether the phone
number or email address belongs to an AT\&T customer, fraud risk,
port number, IP address prefix, NPA, NXX).  We may also have data
on the edges (transaction start and end times, volume of
transaction data).

Usually a single transaction involves only two nodes, but
sometimes more nodes are involved, as in the case of 800 and
credit card calls.

\subsection {Communities of interest among telephone customers}

We construct community of interest graphs for records of
telephone calls as follows:  start from one phone number, add all
calls made to and from that number, and then add a second ring of
calls to and from those numbers.

Node variables can include: bizosity in and out, usage in and out,
800 usage, international usage, frequency in and out, a fraudiness
rating, whether the number is AT\&T customer,  ...  (Add definitions)

Likely edge variables are the time stamp and duration in seconds.

These are things we might be interested in learning about the data:

\begin{itemize} \itemsep 0em
\item Questions on nodes

  \begin{itemize} \itemsep 0em
  \item Which numbers have high bizosity ratings?  Which numbers make
    most of their calls to high bizosity numbers?  Receive their calls
    from high bizosity numbers?
  \item For a given node, identify its strongly connected component. 
    (A strongly connected component is one in which all nodes can
    reach all others by a {\em directed} path.  It can have only one,
    and it can be found by depth first search.)
  \item If we also had fraud data on each node, we would want to look for
    fraudsters that we didn't previously know.  We'd do that by
    looking for low-risk nodes which share an edge with at least one
    high-risk node.
  \item We might also want to identify new chat lines, by looking for
    nodes with overlapping transactions.
  \end {itemize}

\item Questions on edges
  \begin{itemize} \itemsep 0em
  \item Who talks the most often?  Who has the longest calls
  \item Are there groups of calls that cluster by time?
  \end{itemize}

\end{itemize}

We are also interested in comparing pairs of COI graphs.  One
purpose for this comparison is to identify people who have moved:
We'll compare COI data for an account that was just closed
with a new account bearing a similar customer name.  Are there
visual comparisons that can supplement the use of distance
metrics?  We might display all nodes that two graphs have in
common, together with each node that shares an edge with
one of the common nodes.

We are also investigating the use of graph comparison for
identifying subscription wireless fraud.  Once a wireless phone
number has been identified as fraudulent, it is quickly cut off.
If a new number appears shortly thereafter which has similar
behavior, then it too may be fradulent.

\subsection {Netflow data}

The Netflow data is the closest thing we've got to ``call-level''
measurement of IP traffic.  Each flow corresponds to a set of packets
between the same two end-points, measured at the same location, and
occurring close together in time.  By end-points, we mean the same pair
of IP addresses and same port numbers (TCP or UDP connection).  There 
are four levels of hierarchy for each end-point, and this hierarchy
can be used to aggregate the data in different ways:

\begin{itemize} \itemsep 0em
  \item TCP/UDP end-point: IP address and port number (e.g., 1.2.3.4, port 80)
  \item Host: IP address (1.2.3.4)
  \item Network address: IP subnet (1.2.3.0/24; e.g., 1.2.3.*)
  \item Domain: autonomous system number (AS 7018)
\end{itemize}

Each flow (or edge) has a start and end time, and consists
of a certain number of bytes and packets.  Other edge
variables are protocol, type of service, and sequence number.

For each node, we know the IP address, interface number, port,
and autonomous system number (domain) for the IP address,

\subsection {Using netflow data to study the network}


\subsection {Using netflow data to study customer behavior}

Since the AT\&T netflow data includes traffic between Worldnet
customers and web sites, we can use the data in an indirect
fashion to study the relationships between customers and web
sites.

We can build a graph in which vertices are web sites and edges
indicate a relationship between a pair of web sites.  These edges
would be inferred from customer behavior as follows:  We can
identify web sites that are visited by a particular customer
within some fixed time interval -- say, 5 minutes -- to place
an edge between the two vertices.

The definition of an edge in this example is much different than
in the previous examples, since it's inferred rather than
immediately present in the data.  Note also that these edges are
not directed, which is probably unusual for transaction data.

The definition of these edges is a slippery thing, since a web
page may contain multiple URLs, and we don't want to be misled by
such things as advertising servers like Doubleclick and caching
services like Akamai -- this is not data that captures anything
useful about the customer request!

There's ambiguity at the user end, too, since the customer IP
addresses are assigned only for the duration of a particular
session, and one may want to distinguish the sessions of two
different customers, one of whom logs in just as the other logs
out.

It would be interesting to look at relationships between classes
of web hosts, such as bookstores (Amazon, Barnes \& Noble, Borders,
et al).  Do customers move from one to another, doing comparison
shopping?  Do we want to aggregate all bookstores into a single
node?

We also want to identify relationships between unlike hosts -- do
people frequently check the news and then the weather?  Are there
customer trends we could spot in these graphs -- are Worldnet
customers now investigating online banks in greater numbers?

It would be important to be able to erase both the extremely busy
sites (CNN, Amazon, weather.com), since they would clutter and
dominate the graph.  This same problem exists for telephone
calling data (1.800.CALL.ATT, for instance, receives many more
calls than most other numbers!), but it's much worse for web
sites.  We would also want to have ways to avoid the extremely
rarely visited sites.

Each edge could have a distance variable indicating some measure
of the strength of the relationship between the two web sites --
it might be defined by the number of customers who visited both.
The ideal layout algorithm for this data would be something like
multidimensional scaling, which would attempt to position the
vertices such that those distances were replicated on the
screen.

The set of complementary graphs could also be interesting:  in
this case, the nodes would be {\em customer} IP addresses (or
customer ids, if we can make that translation), and the inferred
links would connect customers who visited the same sites.  This
is more obviously analogous to the community-of-interest graphs
for telephone call data described above, in which we can look for
customers with similar interests and behavior.


\section{Layout}

Since we are thinking of the exploratory data analysis on-line,
we are more interested in layout algorithms that are fast than
those that are perfect.  We will probably draw only straight-line
edges, and our layout algorithms probably won't attempt to minimize
edge crossings.


\begin{itemize} \itemsep 0em
\item Layout types
   \begin{itemize} \itemsep 0em
   \item layouts in 2d, 3d, and higher
   \item hierarchical, non-hierarchical layouts
   \item concentric circles (or spheres)
   \item hyperbolic graphs
   \item iterative layouts, like multidimensional scaling in xgvis

   \item constrained layouts
     \begin{itemize} \itemsep 0em
     \item place related nodes near each other: in the
           netflow case, big sources could be close
     \item place nodes on a map
     \end{itemize}
   \end{itemize}

 \item Multiple unconnected graphs: display in the same window?
   Could treat as separate graphs, partition the screen and lay
   out each graph separately.  The screen partitioning may need
   to be fluid, since we don't know a priori how much space each
   graph will need.

 \item Layout interactions
   \begin{itemize} \itemsep 0em
   \item Reset the center node 
   \item Move nodes and edges to adjust the layout
   \end{itemize}

\end{itemize}

\section{Thinning or expanding the graph}

\begin{itemize} \itemsep 0em
\item Thinning
   \begin{itemize} \itemsep 0em
   \item Reset display thresholds
     \begin{itemize} \itemsep 0em
     \item eliminate transactions (edges) based on some edge variable:
           length, size, cost
     \item eliminate nodes based on some node variable: risk, ?
     \end{itemize}
   \item Aggregate
     \begin{itemize} \itemsep 0em
     \item interactively: by selection
     \item logically: by node variable (prefix, NPA) or edge
           variable (duration)
     \item automatically: set thresholds such that if there are
           too many nodes to lay out, automatic aggregation will
           occur.
     \end{itemize}
   \end{itemize}

\item Expansion
   \begin{itemize} \itemsep 0em
   \item Add the transactions for a selected node --
       requires some database connection
   \item Deaggregation: a logical zoom into an aggregate node
   \end{itemize}

\end{itemize}
   
\section {Representation}

\begin{itemize} \itemsep 0em

\item Note: the edge that's drawn represents some number of
  transactions, in different directions, and needs to be
  distinguished from the ``edge'' that exists in the data.

\item Edge direction can be distinguished by drawing the line in
  two sections, using proportional coloring of the line, or
  proportional length and thickness of the two sections.

\item Edge color or thickness can represent

  \begin{itemize} \itemsep 0em
  \item whether the edge is within the current time window
  \item the sum (mean, median) of any edge variable (size, duration)
  \item the value of any edge variable
  \end{itemize}

\item  Node color or size can represent
  \begin{itemize} \itemsep 0em
  \item the number of simultaneous transactions
  \item the node degree (in-degree, out-degree)
  \item the value of any node variable
  \end{itemize}

\end {itemize}


\section{Interaction with the graph}

\begin{itemize} \itemsep 0em
\item Direct manipulation

  \begin{itemize} \itemsep 0em
  \item select a node or set of nodes, an edge or set of edges to
    \begin{itemize} \itemsep 0em
     \item zoom and pan
     \item highlight (paint)
     \item adjust the layout (see layout interactions)
     \item hide, aggregate, expand (see thinning or expanding)
     \item apply some algorithm to the selected set, such as 
           reporting the average transaction size
     \end{itemize}


 \end{itemize}

\item Interaction via menus (indirect manipulation)

  \begin{itemize} \itemsep 0em
   \item ask for a new layout
   \item display some label next to each edge or node,
         changing the type of data to be displayed
  \end{itemize}

\end{itemize}


\section {Linked views}

\begin{itemize} \itemsep 0em

\item scatterplots/scatmats/parcoords plots of node or edge data
  \begin{itemize} \itemsep 0em
  \item use node or edge variables to highlight (or hide) some of
        the graph: big sources, low-traffic edges, ...
  \item note: when linking from a scatterplot of edge data to the
        graph, we might make a translation that painting a point
        with a bigger glyph is equivalent to painting an edge with
        a wider line
  \end{itemize}

\item seecalls view

\item other graphs:
  \begin{itemize} \itemsep 0em
  \item the same nodes in different time periods, linked by node
  \item different nodes in the same time period, possibly linked by time
  \item how would one diff a pair of graphs?  and how would one
        display the result?
  \end{itemize}

\end{itemize}


\section {Time}

 One way to manage time:  lay out the graph for all data, but
   show or highlight the nodes and edges that fall within a time window,
   Scroll or step through the window using a scrollbar or spinner.
   (When edges are aggregates, this may be tricky.)
 
 When we ask "tell me everybody who has called this guy", we might
  like to say "during the current time window" or "in the entire db"

\section {Presentation}

\begin{itemize} \itemsep 0em
\item Generate images for use in browsers -- simple jpegs,
  active images
\item Annotation: labelling graphs is tricky!
\end {itemize}


\section {Implementation}

There are two possibilities we're pursuing in parallel:
implementing graph visualization as a module in ORCA and inside ggobi.


\begin{itemize} \itemsep 0em
\item ORCA  
  \begin{itemize} \itemsep 0em
  \item Pros
    \begin{itemize} \itemsep 0em
    \item appealing programming environment, java
    \item easy linking to a variety of plot types
    \end{itemize}
  \item Cons
    \begin{itemize} \itemsep 0em
    \item slow; just about unusable on fry
    \item no command line interface
    \end{itemize}
  \end{itemize}

\item ggobi
  \begin{itemize} \itemsep 0em
  \item Pros
    \begin{itemize} \itemsep 0em
    \item embeddable in R; command line available
      \begin{itemize} \itemsep 0em
      \item talk to a database 
      \item use algorithms in R to thin the graph, find strongly connected
        components, etc.
      \end{itemize}
    \item attractive C++ graph component library available (GGCL)
    \item easy linking to scatterplot, scatmat, parcoords plot
    \item good performance
    \end{itemize}
  \item Cons
    \begin{itemize} \itemsep 0em
    \item not quite as portable as java code?
    \item headed for obsolence once java gets up to speed?
    \end{itemize}
  \end{itemize}
\end{itemize}
    

\end{document}
