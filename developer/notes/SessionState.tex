\documentclass{article}

\begin{document}
\begin{abstract}
  This is a note on preserving information from one ggobi session to
  another.  It is a start on being able to restart an old session with
  the same configuration.  Some of this is easiest to do in R.
\end{abstract}

We look for a file identified by 
\begin{enumerate}
\item the --init argument,
\item the environment variable GGOBIRC,
\item or the \file{.ggobirc} file 
in the users home directory. (Need to figure out
\end{enumerate}
We read this as an XML file according to the DTD \file{ggobiInit.dtd}.
This file contains
information such as
\begin{description}
\item[previous data files]
\item[displays]
\end{description}

Other information such as color preferences, how many files to
remember can also be added.  Additionally, we can attempt to store
information such as transformations, new variables, rotations, and so
on.  We want these in symbolic form.

Additionally, we can use this file to store information about plugins,
etc.


\section{Writing the Session File}

When do we write this and how do we preserve
existing contents.

\end{document}

