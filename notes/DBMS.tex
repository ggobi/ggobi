\documentclass{article}

\begin{document}

\begin{abstract}
  A common source of data is Database Management Systems (DBMS), and
  specifically relational databases.  We describe the facilities
  currently provided by Ggobi for reading data from MySQL
  (\URL{http://www.mysql.org}), a particular DBMS.
\end{abstract}

The ability to read GGobi input from a single XML opens a large number
of possibilities.  Reading data from a database however is also a
powerful extension to any system.  Databases offer many conveniences
for storing values in comparison with files, URLs, etc.
These advantages include
\begin{itemize}

\item Large quantities of data can be managed by software
 designed for that purpose. 

\item Only a single instance of the data is needed, as applications
  can dynamically query the segment of that data they are interested
  \textsl{when they need it}.  These client applications can be
  written in any of numerous languages that provide a library or
  bindings for the DBMS.  Support for many of these systems is
  provided in Python, Perl, Java, C, etc.

\item Data can be updated dynamically and made available to clients.

\item Updates to the data (e.g. correction of errors) need be
  performed in a single location, simplifying administration
and providing greater data integrity.

\item Many of the common, basic computations can be performed
using SQL specified in a client process and performed locally on the
data within the DBMS.

\item Most DBMS provide a security model for limiting access
to the data, allowing great resolution for the administrators.

\end{itemize}


\section{MySQL}
MySQL is a database management system ...

\section{GGobi \& MySQL}
When one starts GGobi, one can instruct it
to use a connection to a MySQL database
using the \flag{-mysql}.
\begin{verbatim}
 ggobi -mysql
\end{verbatim}
(No argument should be specified after the 
\flag{-mysql} argument.)

In this case, a dialog is displayed that allows one to specify any and
all of the details about the database and the data that one wants.
The fields are described in table \




\section{Default Settings}

\section{Describing the Source of the Data}


\end{document}
