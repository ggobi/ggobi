\documentclass{article}
\usepackage{fullpage}
\usepackage{pstricks}

\hyphenation{XGobi}
\hyphenation{ggobi}
\hyphenation{di-men-sion-al two--di-men-sion-al}
\input epsf
\def\Rcirc{\mbox{\small{\ooalign{\hfil\raise.07ex\hbox{\tiny R}\hfil\crcr\mathhexbox20D}}}}


\title {GGobi manual}

\begin {document}

\section{Introduction}

GGobi is a data visualization system with state-of-the-art
interactive and dynamic methods for the manipulation of views of
data.  It represents a big step in the evolution of XGobi, with
multiple plotting windows, more flexible color management, xml file
handling, and better portability to Windows. 

GGobi has been designed so that it can be embedded in
other software and controlled using an API (application programming
interface).  This design has been developed and tested in partnership
with R.  When GGobi is used with R, the result is a full marriage
between GGobi's direct manipulation graphical environment and R's
familiar extensible environment for statistical data analysis.

It has the same graphical functionality whether it is running
standalone or embedded in other software.  That functionality
includes 2-D displays of projections of points and lines in
high-dimensional spaces, as well as scatterplot matrices and parallel
coordinate displays.  Projection tools include average shifted
histograms of single variables, plots of pairs of variables, and
grand tours of multiple variables.  Views of the data can be
reshaped.  Points can be labeled and brushed with glyphs and colors.
Several displays can be open simultaneously and linked for labeling
and brushing.  Missing data are accommodated and their patterns can
be examined.

\section{Layout and functionality}
\subsection{The major functions}
\subsection{Graphical displays}
\subsection{Variable selection}
\subsection{Control panel}

\section{Data format}

\section{View modes}

\subsection{1D plots}
\subsection{XY plots}
\subsection{Tours}
\subsection{Scaling of axes}
\subsection{Linked brushing of points and lines}
\subsection{Linked identification of points with labels}
\subsection{Line editing}

\section {Display types}
\subsection{Scatterplot}
\subsection{Scatterplot matrix}
\subsection{Parallel coordinate window}
\subsection{Missing data displays}

\section{Tools}
\subsection{Jittering}
\subsection{Imputation controls for missing data}
\subsection{Smooths}
\subsection{Subsetting}

\section{Implementation}
\section{Integration of ggobi in other software systems}

\section{Appendix: Differences from xgobi}

%%\begin {itemize}
%%\item Multiple displays
%%\item Display types
%%\item Data format
%%\item Variable selection
%%\item Changes in ViewModes and Tools
%%\item Color brushing
%%\item 1D tour
%%\item Sphering
%%\item Scaling
%%\item DisplayTree
%%\item Using ggobi with other software
%%\item Help
%%\end {itemize}

\end {document}

